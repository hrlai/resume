% LaTeX Curriculum Vitae Template
%
% Copyright (C) 2004-2009 Jason Blevins <jrblevin@sdf.lonestar.org>
% http://jblevins.org/projects/cv-template/
%
% You may use use this document as a template to create your own CV
% and you may redistribute the source code freely. No attribution is
% required in any resulting documents. I do ask that you please leave
% this notice and the above URL in the source code if you choose to
% redistribute this file.

\documentclass[a4paper]{article}

\usepackage{hyperref}
\usepackage{geometry}
% \usepackage{paralist}  % compact list
\usepackage{multicol}
\usepackage{enumitem}
\setlist{nosep,after=\vspace{\baselineskip}}

% Comment the following lines to use the default Computer Modern font
% instead of the Palatino font provided by the mathpazo package.
% Remove the 'osf' bit if you don't like the old style figures.
% \usepackage[T1]{fontenc}
% \usepackage[sc,osf]{mathpazo}

% Set your name here
\def\name{Hao Ran LAI}

% Replace this with a link to your CV if you like, or set it empty
% (as in \def\footerlink{}) to remove the link in the footer:
\def\footerlink{}

% The following metadata will show up in the PDF properties
\hypersetup{
  colorlinks = true,
  urlcolor = black,
  pdfauthor = {\name},
  pdfkeywords = {ecology, statistics, functional trait, forest},
  pdftitle = {\name: Curriculum Vitae},
  pdfsubject = {Curriculum Vitae},
  pdfpagemode = UseNone
}

\geometry{
  left=1.0in,
  right=1.0in,
  top=1.0in,
  bottom=1.0in
}

% Customise tabular spacing (padding)
\renewcommand{\arraystretch}{1.5}
\setlist{noitemsep, nosep, topsep=0pt, partopsep=0pt, parsep=0pt, after=\vspace{-\baselineskip}}

% Customize page headers
\pagestyle{myheadings}
\markright{\name}
\thispagestyle{empty}

% Custom section fonts
\usepackage{sectsty}
\sectionfont{\rmfamily\mdseries\Large}
\subsectionfont{\rmfamily\mdseries\itshape\large}

% Other possible font commands include:
% \ttfamily for teletype,
% \sffamily for sans serif,
% \bfseries for bold,
% \scshape for small caps,
% \normalsize, \large, \Large, \LARGE sizes.

% Don't indent paragraphs.
\setlength\parindent{0em}

% Use dash as bullets
\renewcommand\labelitemi{---}

\begin{document}

% Place name at left
{\huge \name}

% Alternatively, print name centered and bold:
%\centerline{\huge \bf \name}

\vspace{0.25in}

\begin{minipage}{0.45\linewidth}
  Centre for Integrative Ecology \\
  School of Biological Sciences \\
  Te Kura Pūtaiao Koiora  \\
  University of Canterbury \\
  Christchurch 8140 \\
  New Zealand
\end{minipage}
\begin{minipage}{0.45\linewidth}
  \begin{tabular}{ll}
    Tel:   & (+64) 27 307 6493 \\
    Email: & \href{mailto:hrlai.ecology@gmail.com}{\tt hrlai.ecology@gmail.com} \\
  \end{tabular}
\end{minipage}



\section*{Education}

\begin{tabular}{p{0.13\linewidth} p{0.87\linewidth}}

2015--2018 & Ph.D., Ecology and Evolutionary Biology, National University of Singapore. Advisors: Michiel van Breugel \& Hugh Tan Tiang Wah. Thesis: \textit{Functional ecology of tropical secondary forests} \\

2009--2012 & B.Sc. (Hons.) Ecology, University of Queensland. Advisor: Margaret M. Mayfield. Thesis: \textit{Functional recovery following logging in subtropical forests}\\

\end{tabular}


\section*{Employment}

\begin{tabular}{p{0.13\linewidth} p{0.87\linewidth}}

2019--current & Research Fellow, Centre for Integrative Ecology, School of Biological Sciences, University of Canterbury, New Zealand
\begin{itemize} 
\item Higher-order species interactions
\end{itemize} \\

2019 & Visiting Research Fellow, Centre for Urban Greenery and Ecology, National Parks Board, Singapore
\begin{itemize} 
\item Tree diameter growth in the urban areas of Singapore
\end{itemize} \\

2018--2019 & Research assistant, Department of Biological Sciences, National University of Singapore, Singapore
\begin{itemize} 
\item Data analysis and plant community survey in the Nee Soon Freshwater Swamp forest, Singapore
\end{itemize} \\

2015--2018 & Teaching assistant, National University of Singapore, Singapore
\begin{itemize} 
\item Biostatistics, Ecology, Field Studies, Plant Biology, and Horticulture
\end{itemize} \\

2015 & Biology Olympiad trainer, National Junior College, Singapore \\

2012--2014 & Research assistant, Mayfield Plant Ecology Lab, University of Queensland, Australia 
\begin{itemize}
\item Survey and analyses of natural plant communities in Queensland subtropical forests and Western Australia grassland 
\end{itemize} \\

2012--2014 & Research officer, Centre for Mined Land Rehabilitation, University of Queensland, Australia
\begin{itemize} 
\item Map and identify threats of coal mining to an endangered plant species
\end{itemize} \\

2012--2014 & Research assistant, Buckley Ecology Lab, University of Queensland, Australia \\

2012--2014 & Course Tutor, University of Queensland, Australia
\begin{itemize} 
\item Biostatistics, Ecology, and Plant Biology
\end{itemize} \\

\end{tabular}


\section*{Publications}

% \subsection*{Journal Articles}
\begin{tabular}{p{\linewidth}}

\hangindent=1cm \underline{Lai, H. R}., Chong, K. Y., Yee, A. T. K., Tan, H. T. W., van. Breugel, M. (2020). Functional traits that moderate tropical tree recruitment during post-windstorm secondary succession. \textit{Journal of Ecology}, 108, 1322--1333. \\

\hangindent=1cm Chiam, Z., Song, X. P., \underline{Lai, H. R}., Tan, H. T. W. (2019). Particulate matter mitigation via plants: Understanding complex relationships with leaf traits. \textit{Science of the Total Environment}, 688, 398--408.\\

\hangindent=1cm Yee, A. T. K., \underline{Lai, H. R}, Chong, K. Y., Neo, L., Koh, C. Y., Tan, S. Y., ... Tan, H. T. W. (2019). Short-term responses in a secondary tropical forest after a severe windstorm event. \textit{Journal of Vegetation Science}, 30, 720--731. \\

\hangindent=1cm van Breugel, M., Craven D., \underline{Lai, H. R}, Baillon M., Turner, B.L., Hall, J.S. (2019). Soil nutrients and dispersal limitation shape compositional variation in secondary tropical forests across multiple scales. \textit{Journal of Ecology}, 107, 566--581. \textit{Special Feature -- Ecological succession in a changing world}.\\

\hangindent=1cm Wainwright, C. E., HilleRisLambers, J., \underline{Lai, H. R.}, Loy, X., Mayfield, M. M. (2019). Distinct responses of niche and fitness differences to water availability underlie variable coexistence outcomes in semi-arid annual plant communities. \textit{Journal of Ecology}, 107, 293--306.\\

\hangindent=1cm Lam, W.N., \underline{Lai, H.R.}, Lee. C., Tan, H.T.W. (2018) Evidence for pitcher trait-mediated coexistence between sympatric \textit{Nepenthes} pitcher plant species across geographical scales. \textit{Plant Ecology and Diversity}, 11(3), 283--294.\\

\hangindent=1cm \underline{Lai, H.R.}, Hall, J.S., Batterman, S.A., Turner, B.L., van Breugel, M. (2018). Nitrogen fixer abundance has no effect on biomass recovery during tropical secondary forest succession. \textit{Journal of Ecology}, 106, 1415--1427. \textit{Special Feature -- Linking organismal functions, life history strategies and population performance}. \\

\hangindent=1cm Bimler, M.D., Stouffer, D.B., \underline{Lai, H.R.}, Mayfield, M.M. (2018). Accurate predictions of coexistence in natural systems require the inclusion of facilitative interactions and environmental dependency. \textit{Journal of Ecology}, 106(5), 1839--1852. \textit{Special Feature -- Biotic controls of plant coexistence}.\\

\hangindent=1cm Wainwright, C.E., Staples, T.L., Charles, L.S., Flanagan, T.C., \underline{Lai, H.R.}, Loy, X., Reynolds, V.A., Mayfield, M.M. (2018). Links between community ecology theory and ecological restoration are on the rise. \textit{Journal of Applied Ecology}, 55, 570--581.\\

\hangindent=1cm Sams, M.A., \underline{Lai, H.R.}, Bonser, S.P., Vesk, P.A., Kooyman, R.M., Metcalfe, D.J., Morgan, J.W., Mayfield, M.M. (2017). Landscape context explains changes in the functional diversity of regenerating forests better than climate or species richness. \textit{Global Ecology and Biogeography}, 26, 1165--1176. \\

\hangindent=1cm \underline{Lai, H.R.}, Hall, J.S., Turner, B.L., van Breugel, M. (2017). Liana effects on biomass dynamics strengthen during secondary forest succession. \textit{Ecology}, 98, 1062--1070. \\

\hangindent=1cm \underline{Lai, H.R.}, Mayfield, M.M., Gay-des-combes, J.M., Spiegelberger, T., Dwyer, J.M. (2015). Distinct invasion strategies operating within a natural annual plant system. \textit{Ecology Letters}, 18, 336--346. \\

\end{tabular}

\section*{Conferences \& Presentations}
\begin{tabular}{p{0.13\linewidth} p{0.87\linewidth}}
2020 & School of Biological Sciences Seminar. University of Canterbury, New Zealand. \\
2020 & International Statistical Ecology Conference (ISEC). virtual. \\
2019 & Centre for Urban Greenery and Ecology, National Parks Board, Singapore. \\
2018 & Association of Tropical Biology and Conservation (ATBC). Kuching, Malaysia. \\
2016 & Conservation Asia. Singapore. \\
2015 & 20th Biological Science Graduate Congress. Bangkok, Thailand. \\
\end{tabular}


\section*{Awards \& Scholarships}
\begin{tabular}{p{0.13\linewidth} p{0.87\linewidth}}
2019 & International Society for Plant Molecular Biology Medal for most outstanding thesis \\
2019 & CUGE Visiting Research Fellowship, National Parks Board Singapore \\
2017 \& 2018 & Teaching Assistant Award, National University of Singapore \\
2016 & City Developments Limited (CDL) Urban Ecology and Conservation Scholarship \\
2013 & University Medal, University of Queensland \\
2011 & Summer Research Scholarship, University of Queensland \\
2010 & D.A. Herbert Prize in Botany \& F.A Perkins Prize in Entomology, University of Queensland 
\end{tabular}


\section*{Services \& Outreach}
\begin{tabular}{p{0.13\linewidth} p{0.87\linewidth}}
2020 & \texttt{RMarkdown} tutorial, University of Canterbury, New Zealand \\
2017 & Science Research Programme, Hwa Chong Institution, Singapore \\
2016 & Festival of Biodiversity, Singapore Botanic Gardens \\
2012--2014 & Volunteer, National Parks Association of Queensland Inc. (NPAQ) \\
Ongoing & Reviewer for \textit{Journal of Ecology}, \textit{Ecology}, \textit{Methods in Ecology and Evolution}, \textit{Ecological Applications}, \textit{Biodiversity and Conservation}, \textit{Biological Invasions}, \textit{Science of the Total Environment}\\
\end{tabular}


\section*{Languages \& Qualifications}
\begin{tabular}{p{\linewidth}}
English (professional working proficiency) \\
Mandarin (mother tongue) and Cantonese (regional dialect) \\
Malay (national language) \\
Malaysian and Singaporean driver's licenses \\
\end{tabular}


\section*{Relevant skills}
\begin{multicols}{2}
\begin{itemize}
	\item \texttt{R} statistical and programming language
	\item \LaTeX~and \texttt{Markdown}
	\item GitHub
\end{itemize}
\columnbreak
\begin{itemize}
	\item Plant survey and identification
	\item Geographic information system
\end{itemize}
\end{multicols}


\section*{Professional references}
\begin{tabular}{p{0.33\linewidth} p{0.33\linewidth} p{0.33\linewidth}}
A/P Daniel B. Stouffer \newline Current supervisor \newline Associate Professor, University of Canterbury, New Zealand \newline Tel: (+64) 3 369 2880 \newline Email: \url{daniel.stouffer@canterbury.ac.nz} &
Dr Michiel van Bruegel \newline Former PhD supervisor \newline Assistant Professor, Yale--NUS College, Singapore \newline Tel: (+65) 6601 3705 \newline Email: \url{michiel.vanbreugel@yale-nus.edu.sg} &
A/P Hugh Tiang Wah Tan \newline Former PhD co-supervisor \newline Associate Professor, National University of Singapore \newline Tel: (+65) 6516 2708 \newline Email: \url{hughtan@nus.edu.sg} 
%Dr. Kwek Yan Chong & Former employer / current collaborator \newline Senior Tutor, National University of Singapore \newline Tel: (+65) 6779 2486  \newline Email: \url{dbscky@nus.edu.sg}
\end{tabular}

\vfill

% Footer
\begin{center}
  \begin{footnotesize}
    Last updated: \today \\
    \href{\footerlink}{\texttt{\footerlink}}
  \end{footnotesize}
\end{center}

\end{document}
